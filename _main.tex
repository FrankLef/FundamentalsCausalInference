% Options for packages loaded elsewhere
\PassOptionsToPackage{unicode}{hyperref}
\PassOptionsToPackage{hyphens}{url}
%
\documentclass[
]{book}
\usepackage{amsmath,amssymb}
\usepackage{lmodern}
\usepackage{ifxetex,ifluatex}
\ifnum 0\ifxetex 1\fi\ifluatex 1\fi=0 % if pdftex
  \usepackage[T1]{fontenc}
  \usepackage[utf8]{inputenc}
  \usepackage{textcomp} % provide euro and other symbols
\else % if luatex or xetex
  \usepackage{unicode-math}
  \defaultfontfeatures{Scale=MatchLowercase}
  \defaultfontfeatures[\rmfamily]{Ligatures=TeX,Scale=1}
\fi
% Use upquote if available, for straight quotes in verbatim environments
\IfFileExists{upquote.sty}{\usepackage{upquote}}{}
\IfFileExists{microtype.sty}{% use microtype if available
  \usepackage[]{microtype}
  \UseMicrotypeSet[protrusion]{basicmath} % disable protrusion for tt fonts
}{}
\makeatletter
\@ifundefined{KOMAClassName}{% if non-KOMA class
  \IfFileExists{parskip.sty}{%
    \usepackage{parskip}
  }{% else
    \setlength{\parindent}{0pt}
    \setlength{\parskip}{6pt plus 2pt minus 1pt}}
}{% if KOMA class
  \KOMAoptions{parskip=half}}
\makeatother
\usepackage{xcolor}
\IfFileExists{xurl.sty}{\usepackage{xurl}}{} % add URL line breaks if available
\IfFileExists{bookmark.sty}{\usepackage{bookmark}}{\usepackage{hyperref}}
\hypersetup{
  pdftitle={Study project: undamentals of Causal Inferences With R},
  pdfauthor={François Lefebvre},
  hidelinks,
  pdfcreator={LaTeX via pandoc}}
\urlstyle{same} % disable monospaced font for URLs
\usepackage{color}
\usepackage{fancyvrb}
\newcommand{\VerbBar}{|}
\newcommand{\VERB}{\Verb[commandchars=\\\{\}]}
\DefineVerbatimEnvironment{Highlighting}{Verbatim}{commandchars=\\\{\}}
% Add ',fontsize=\small' for more characters per line
\usepackage{framed}
\definecolor{shadecolor}{RGB}{248,248,248}
\newenvironment{Shaded}{\begin{snugshade}}{\end{snugshade}}
\newcommand{\AlertTok}[1]{\textcolor[rgb]{0.94,0.16,0.16}{#1}}
\newcommand{\AnnotationTok}[1]{\textcolor[rgb]{0.56,0.35,0.01}{\textbf{\textit{#1}}}}
\newcommand{\AttributeTok}[1]{\textcolor[rgb]{0.77,0.63,0.00}{#1}}
\newcommand{\BaseNTok}[1]{\textcolor[rgb]{0.00,0.00,0.81}{#1}}
\newcommand{\BuiltInTok}[1]{#1}
\newcommand{\CharTok}[1]{\textcolor[rgb]{0.31,0.60,0.02}{#1}}
\newcommand{\CommentTok}[1]{\textcolor[rgb]{0.56,0.35,0.01}{\textit{#1}}}
\newcommand{\CommentVarTok}[1]{\textcolor[rgb]{0.56,0.35,0.01}{\textbf{\textit{#1}}}}
\newcommand{\ConstantTok}[1]{\textcolor[rgb]{0.00,0.00,0.00}{#1}}
\newcommand{\ControlFlowTok}[1]{\textcolor[rgb]{0.13,0.29,0.53}{\textbf{#1}}}
\newcommand{\DataTypeTok}[1]{\textcolor[rgb]{0.13,0.29,0.53}{#1}}
\newcommand{\DecValTok}[1]{\textcolor[rgb]{0.00,0.00,0.81}{#1}}
\newcommand{\DocumentationTok}[1]{\textcolor[rgb]{0.56,0.35,0.01}{\textbf{\textit{#1}}}}
\newcommand{\ErrorTok}[1]{\textcolor[rgb]{0.64,0.00,0.00}{\textbf{#1}}}
\newcommand{\ExtensionTok}[1]{#1}
\newcommand{\FloatTok}[1]{\textcolor[rgb]{0.00,0.00,0.81}{#1}}
\newcommand{\FunctionTok}[1]{\textcolor[rgb]{0.00,0.00,0.00}{#1}}
\newcommand{\ImportTok}[1]{#1}
\newcommand{\InformationTok}[1]{\textcolor[rgb]{0.56,0.35,0.01}{\textbf{\textit{#1}}}}
\newcommand{\KeywordTok}[1]{\textcolor[rgb]{0.13,0.29,0.53}{\textbf{#1}}}
\newcommand{\NormalTok}[1]{#1}
\newcommand{\OperatorTok}[1]{\textcolor[rgb]{0.81,0.36,0.00}{\textbf{#1}}}
\newcommand{\OtherTok}[1]{\textcolor[rgb]{0.56,0.35,0.01}{#1}}
\newcommand{\PreprocessorTok}[1]{\textcolor[rgb]{0.56,0.35,0.01}{\textit{#1}}}
\newcommand{\RegionMarkerTok}[1]{#1}
\newcommand{\SpecialCharTok}[1]{\textcolor[rgb]{0.00,0.00,0.00}{#1}}
\newcommand{\SpecialStringTok}[1]{\textcolor[rgb]{0.31,0.60,0.02}{#1}}
\newcommand{\StringTok}[1]{\textcolor[rgb]{0.31,0.60,0.02}{#1}}
\newcommand{\VariableTok}[1]{\textcolor[rgb]{0.00,0.00,0.00}{#1}}
\newcommand{\VerbatimStringTok}[1]{\textcolor[rgb]{0.31,0.60,0.02}{#1}}
\newcommand{\WarningTok}[1]{\textcolor[rgb]{0.56,0.35,0.01}{\textbf{\textit{#1}}}}
\usepackage{longtable,booktabs,array}
\usepackage{calc} % for calculating minipage widths
% Correct order of tables after \paragraph or \subparagraph
\usepackage{etoolbox}
\makeatletter
\patchcmd\longtable{\par}{\if@noskipsec\mbox{}\fi\par}{}{}
\makeatother
% Allow footnotes in longtable head/foot
\IfFileExists{footnotehyper.sty}{\usepackage{footnotehyper}}{\usepackage{footnote}}
\makesavenoteenv{longtable}
\usepackage{graphicx}
\makeatletter
\def\maxwidth{\ifdim\Gin@nat@width>\linewidth\linewidth\else\Gin@nat@width\fi}
\def\maxheight{\ifdim\Gin@nat@height>\textheight\textheight\else\Gin@nat@height\fi}
\makeatother
% Scale images if necessary, so that they will not overflow the page
% margins by default, and it is still possible to overwrite the defaults
% using explicit options in \includegraphics[width, height, ...]{}
\setkeys{Gin}{width=\maxwidth,height=\maxheight,keepaspectratio}
% Set default figure placement to htbp
\makeatletter
\def\fps@figure{htbp}
\makeatother
\setlength{\emergencystretch}{3em} % prevent overfull lines
\providecommand{\tightlist}{%
  \setlength{\itemsep}{0pt}\setlength{\parskip}{0pt}}
\setcounter{secnumdepth}{5}
\usepackage{booktabs}
\ifluatex
  \usepackage{selnolig}  % disable illegal ligatures
\fi
\usepackage[]{natbib}
\bibliographystyle{plainnat}

\title{Study project: undamentals of Causal Inferences With R}
\author{François Lefebvre}
\date{2022-02-01}

\usepackage{amsthm}
\newtheorem{theorem}{Theorem}[chapter]
\newtheorem{lemma}{Lemma}[chapter]
\newtheorem{corollary}{Corollary}[chapter]
\newtheorem{proposition}{Proposition}[chapter]
\newtheorem{conjecture}{Conjecture}[chapter]
\theoremstyle{definition}
\newtheorem{definition}{Definition}[chapter]
\theoremstyle{definition}
\newtheorem{example}{Example}[chapter]
\theoremstyle{definition}
\newtheorem{exercise}{Exercise}[chapter]
\theoremstyle{definition}
\newtheorem{hypothesis}{Hypothesis}[chapter]
\theoremstyle{remark}
\newtheorem*{remark}{Remark}
\newtheorem*{solution}{Solution}
\begin{document}
\maketitle

{
\setcounter{tocdepth}{1}
\tableofcontents
}
\hypertarget{study-projec}{%
\chapter*{Study Projec}\label{study-projec}}
\addcontentsline{toc}{chapter}{Study Projec}

\hypertarget{about}{%
\section*{About}\label{about}}
\addcontentsline{toc}{section}{About}

This is a study project of \citet{brumback2022}. Many thanks to Babette Brumback for a
great book to tackle the question of causality. It is \emph{extremely} in my
professional life which is not about making predictions but rather come up
with intervention plans (business intel).

The suggestions for errata are in the section \emph{Errata}. Comments are also
included in the section \emph{Comments}.

\hypertarget{packages}{%
\section*{Packages}\label{packages}}
\addcontentsline{toc}{section}{Packages}

The functions have been rewritten to simplify them and improve the
learning experience (personal opinion) for newbies such as me.

The following packages are so useful that they
could not be avoided. In particular, \texttt{dplyr} and \texttt{tidyr} make the coding
experience so much more interesting that one could almost claim they have become
the standard in \texttt{R} coding.

\begin{itemize}
\tightlist
\item
  \texttt{dplyr} for data wrangling. see \citet{R-dplyr}
\item
  \texttt{tidyr} for data wrangling. see \citet{R-tidyr}
\item
  \texttt{ggplot2} for plots, see \citet{R-ggplot2}`
\item
  \texttt{gt} for all tables, see \citet{R-gt}
\item
  \texttt{dagitty} for analysis of structural causal models, see \citet{R-dagitty}
\item
  \texttt{ggdag} for directed acyclic graphs, see \citet{R-ggdag}
\item
  \texttt{gee}: a generalized estimation equation solver. Introduced in chapter 4.
  See \citet{R-gee}.
\item
  \texttt{MonteCarlo}: To perform Monte Carlo simulations. Introduced in chapter 4,
  section 4.2. See \citet{R-MonteCarlo}.
\item
  \texttt{simstudy}: To run all sorts of simulations. A great tool to learn right from the
  start since simulations are so important. See \citet{R-simstudy}.
\end{itemize}

\hypertarget{errata}{%
\chapter*{Errata}\label{errata}}
\addcontentsline{toc}{chapter}{Errata}

\hypertarget{preface}{%
\section*{Preface}\label{preface}}
\addcontentsline{toc}{section}{Preface}

page xi, last word of first paragraph is \textbf{standaridzation},
s/b \emph{standardization}

\hypertarget{chapter-1}{%
\section*{Chapter 1}\label{chapter-1}}
\addcontentsline{toc}{section}{Chapter 1}

\hypertarget{section-1.2.3.2-p.-11}{%
\subsection*{Section 1.2.3.2, p.~11}\label{section-1.2.3.2-p.-11}}
\addcontentsline{toc}{subsection}{Section 1.2.3.2, p.~11}

The sentence of the 6th line on top of the page
is ``We simulated the data according to the \textbf{hyothetical}'', s/b \emph{hypothetical}

\hypertarget{chapter-2}{%
\section*{Chapter 2}\label{chapter-2}}
\addcontentsline{toc}{section}{Chapter 2}

\hypertarget{figure-2.1-p.-30}{%
\subsection*{Figure 2.1, p.~30}\label{figure-2.1-p.-30}}
\addcontentsline{toc}{subsection}{Figure 2.1, p.~30}

This is really a small detail. The caption of the bottom plot is
\(\hat{E_{np}}(Y \mid A= 1, H =1, T = 1)\), s/b \(\hat{E}_{np}\)

\hypertarget{chapter-3}{%
\section*{Chapter 3}\label{chapter-3}}
\addcontentsline{toc}{section}{Chapter 3}

\hypertarget{typography-section-3.2-p.-40-equation-3.1}{%
\subsection*{Typography: section 3.2 p.~40, equation 3.1}\label{typography-section-3.2-p.-40-equation-3.1}}
\addcontentsline{toc}{subsection}{Typography: section 3.2 p.~40, equation 3.1}

The current latex expression of conditional independence used seems to be
\texttt{(Y(0),\ Y(1))\ \textbackslash{}\ \textbackslash{}text\{II\}\ \textbackslash{}\ T} with the output

\[
(Y(0), Y(1)) \ \text{II} \ T
\]

a better typography would be \texttt{\textbackslash{}perp\textbackslash{}!\textbackslash{}!\textbackslash{}!\textbackslash{}perp} for the symbol \(\perp\!\!\!\perp\).
When used for equation 3.1 as \texttt{(Y(0),\ Y(1))\ \textbackslash{}perp\textbackslash{}!\textbackslash{}!\textbackslash{}!\textbackslash{}perp\ T} we obtain

\[
(Y(0), Y(1)) \perp\!\!\!\perp T
\]

\hypertarget{comments}{%
\chapter*{Comments}\label{comments}}
\addcontentsline{toc}{chapter}{Comments}

\hypertarget{chapter-2-1}{%
\section*{Chapter 2}\label{chapter-2-1}}
\addcontentsline{toc}{section}{Chapter 2}

\hypertarget{section-2.4-p.-31}{%
\subsection*{section 2.4 p.~31}\label{section-2.4-p.-31}}
\addcontentsline{toc}{subsection}{section 2.4 p.~31}

The second sentence of the last paragraph on p.~33 says

\begin{quote}
We also need the \texttt{car} package in order for the summary() function to
operate on boot objects the way we describe.
\end{quote}

This sentence is \textbf{not required} if we use the \texttt{boot::boot.ci()}
which simplifies \texttt{lmodboot.r()} and does not require the \texttt{car} package.
See the code in this document for \texttt{lmodboot.r} in chapter 2.

\hypertarget{chapter-4}{%
\section*{Chapter 4}\label{chapter-4}}
\addcontentsline{toc}{section}{Chapter 4}

\hypertarget{section-4.1}{%
\subsection*{Section 4.1}\label{section-4.1}}
\addcontentsline{toc}{subsection}{Section 4.1}

See the plots in section 4.2. They could be helpful to visualize the changes
in effect measures from one level of modifier to the other.

\hypertarget{section-4.2}{%
\subsection*{Section 4.2}\label{section-4.2}}
\addcontentsline{toc}{subsection}{Section 4.2}

\hypertarget{monte-carlo-simulation}{%
\subsubsection*{Monte Carlo Simulation}\label{monte-carlo-simulation}}
\addcontentsline{toc}{subsubsection}{Monte Carlo Simulation}

A Monte Carlo is provided in section 4.2 and coded in a function
called \texttt{betasim\_effect\_measures()}. It uses the \(Beta\) distribution. It is
helpful in that it

\begin{itemize}
\tightlist
\item
  confirms the same results as in \citet{shanninbrumback2021}
\item
  is less CPU intensive as it needs only 5000 iterations to confirm \citet{shanninbrumback2021}
\item
  is easier to code than \texttt{java} and uses \texttt{R} which is the declared language of
  \citet{brumback2022}
\item
  allows some extra flexibility with the shape parameters of \(Beta\) to investigate
  the conclusion with diffferent curves. See the suggestion for applications below.
\end{itemize}

\hypertarget{page-72-figure-4.1}{%
\subsubsection*{page 72, Figure 4.1}\label{page-72-figure-4.1}}
\addcontentsline{toc}{subsubsection}{page 72, Figure 4.1}

\begin{quote}
The probabilites shown in the Venn diagram do not add up to 100\% because,
for example, the event that RR changes in the same direction as RD but not in
the same direction as the other two measures {[}\ldots{]}. It would akward to
arbitrarily one of those 2 chances as zero.
\end{quote}

\citet{shanninbrumback2021} mentions that it is the result of \emph{not mutually
exclusive events}. That is true. Yet, these events, properly grouped are actually
mutually exclusive. In section 4.2 they are called \textbf{Opposite pairwise events}.
Using these definitions then yes, they are mutually exclusive but
cannot be properly shown in the Venn diagram. This can be easily solved
by splitting the probabilities. See section 4.2 for details.

The end result a proper partitioning of the sample space \(\Omega\) and is,
in fact, a \(\sigma-field\) (See \citet{grimmett}, section 1.2). Yet it does not change the
conclusions reached in \citet{shanninbrumback2021}. Actually, it reinforces them as
this point is \textbf{extremely important} when using probabilities and statistics.

\hypertarget{applications}{%
\subsubsection*{Applications}\label{applications}}
\addcontentsline{toc}{subsubsection}{Applications}

See my sub-section 4.2 called \emph{Applications} where 2 possible
applications are mentioned.

\begin{itemize}
\tightlist
\item
  Data pre-processing (data cleaning)
\item
  Bayesian prior for Beta-binomial model
\end{itemize}

\hypertarget{exercises}{%
\subsection*{Exercises}\label{exercises}}
\addcontentsline{toc}{subsection}{Exercises}

\hypertarget{exercise-1}{%
\subsubsection*{Exercise 1}\label{exercise-1}}
\addcontentsline{toc}{subsubsection}{Exercise 1}

Using the causal power, the conclusion is different than the official answer. It
is not obvious why the official solution does not make use of the \emph{causal power}.

\hypertarget{exercise-5}{%
\subsubsection*{Exercise 5}\label{exercise-5}}
\addcontentsline{toc}{subsubsection}{Exercise 5}

The official solution uses \texttt{gee} with the default family, that is \texttt{gaussian}.

Since the outcome \(attend\) is binary isn't it better to use the \texttt{binomial}
family?

We quote p.~50 from chapter 3 in that respect

\begin{quote}
Because our outcome is binary, we choose to fit the logistic parametric
model
\end{quote}

\hypertarget{chapter-5}{%
\section*{Chapter 5}\label{chapter-5}}
\addcontentsline{toc}{section}{Chapter 5}

The \texttt{dagitty} and \texttt{ggdag} are used extensively.

\hypertarget{hello-bookdown}{%
\chapter{Hello bookdown}\label{hello-bookdown}}

All chapters start with a first-level heading followed by your chapter title, like the line above. There should be only one first-level heading (\texttt{\#}) per .Rmd file.

\hypertarget{setup}{%
\section{Setup}\label{setup}}

Make sure you tell GitHub that the web site is not to be build via Jekyll,
since the \textbf{bookdown} HTML output is already a standalone website. See
section 6.3 of \href{https://bookdown.org/yihui/bookdown/github.html}{bookdown} for
details.

\begin{Shaded}
\begin{Highlighting}[]
\CommentTok{\# create a hidden file .nojekyll}
\CommentTok{\# to tell GitHub that the website is not to be build via Jekyll}
\NormalTok{a\_file }\OtherTok{\textless{}{-}} \FunctionTok{file.path}\NormalTok{(}\FunctionTok{getwd}\NormalTok{(), }\StringTok{".nojekyll"}\NormalTok{)}
\ControlFlowTok{if}\NormalTok{ (}\SpecialCharTok{!}\FunctionTok{file.exists}\NormalTok{(a\_file)) }\FunctionTok{file.create}\NormalTok{(a\_file)}
\end{Highlighting}
\end{Shaded}

\hypertarget{a-section}{%
\section{A section}\label{a-section}}

All chapter sections start with a second-level (\texttt{\#\#}) or higher heading followed by your section title, like the sections above and below here. You can have as many as you want within a chapter.

\hypertarget{an-unnumbered-section}{%
\subsection*{An unnumbered section}\label{an-unnumbered-section}}
\addcontentsline{toc}{subsection}{An unnumbered section}

Chapters and sections are numbered by default. To un-number a heading, add a \texttt{\{.unnumbered\}} or the shorter \texttt{\{-\}} at the end of the heading, like in this section.

\hypertarget{cross}{%
\chapter{Cross-references}\label{cross}}

Cross-references make it easier for your readers to find and link to elements in your book.

\hypertarget{chapters-and-sub-chapters}{%
\section{Chapters and sub-chapters}\label{chapters-and-sub-chapters}}

There are two steps to cross-reference any heading:

\begin{enumerate}
\def\labelenumi{\arabic{enumi}.}
\tightlist
\item
  Label the heading: \texttt{\#\ Hello\ world\ \{\#nice-label\}}.

  \begin{itemize}
  \tightlist
  \item
    Leave the label off if you like the automated heading generated based on your heading title: for example, \texttt{\#\ Hello\ world} = \texttt{\#\ Hello\ world\ \{\#hello-world\}}.
  \item
    To label an un-numbered heading, use: \texttt{\#\ Hello\ world\ \{-\#nice-label\}} or \texttt{\{\#\ Hello\ world\ .unnumbered\}}.
  \end{itemize}
\item
  Next, reference the labeled heading anywhere in the text using \texttt{\textbackslash{}@ref(nice-label)}; for example, please see Chapter \ref{cross}.

  \begin{itemize}
  \tightlist
  \item
    If you prefer text as the link instead of a numbered reference use: \protect\hyperlink{cross}{any text you want can go here}.
  \end{itemize}
\end{enumerate}

\hypertarget{captioned-figures-and-tables}{%
\section{Captioned figures and tables}\label{captioned-figures-and-tables}}

Figures and tables \emph{with captions} can also be cross-referenced from elsewhere in your book using \texttt{\textbackslash{}@ref(fig:chunk-label)} and \texttt{\textbackslash{}@ref(tab:chunk-label)}, respectively.

See Figure \ref{fig:nice-fig}.

\begin{Shaded}
\begin{Highlighting}[]
\FunctionTok{par}\NormalTok{(}\AttributeTok{mar =} \FunctionTok{c}\NormalTok{(}\DecValTok{4}\NormalTok{, }\DecValTok{4}\NormalTok{, .}\DecValTok{1}\NormalTok{, .}\DecValTok{1}\NormalTok{))}
\FunctionTok{plot}\NormalTok{(pressure, }\AttributeTok{type =} \StringTok{\textquotesingle{}b\textquotesingle{}}\NormalTok{, }\AttributeTok{pch =} \DecValTok{19}\NormalTok{)}
\end{Highlighting}
\end{Shaded}

\begin{figure}

{\centering \includegraphics[width=0.8\linewidth]{_main_files/figure-latex/nice-fig-1} 

}

\caption{Here is a nice figure!}\label{fig:nice-fig}
\end{figure}

Don't miss Table \ref{tab:nice-tab}.

\begin{Shaded}
\begin{Highlighting}[]
\NormalTok{knitr}\SpecialCharTok{::}\FunctionTok{kable}\NormalTok{(}
  \FunctionTok{head}\NormalTok{(pressure, }\DecValTok{10}\NormalTok{), }\AttributeTok{caption =} \StringTok{\textquotesingle{}Here is a nice table!\textquotesingle{}}\NormalTok{,}
  \AttributeTok{booktabs =} \ConstantTok{TRUE}
\NormalTok{)}
\end{Highlighting}
\end{Shaded}

\begin{table}

\caption{\label{tab:nice-tab}Here is a nice table!}
\centering
\begin{tabular}[t]{rr}
\toprule
temperature & pressure\\
\midrule
0 & 0.0002\\
20 & 0.0012\\
40 & 0.0060\\
60 & 0.0300\\
80 & 0.0900\\
\addlinespace
100 & 0.2700\\
120 & 0.7500\\
140 & 1.8500\\
160 & 4.2000\\
180 & 8.8000\\
\bottomrule
\end{tabular}
\end{table}

\hypertarget{parts}{%
\chapter{Parts}\label{parts}}

You can add parts to organize one or more book chapters together. Parts can be inserted at the top of an .Rmd file, before the first-level chapter heading in that same file.

Add a numbered part: \texttt{\#\ (PART)\ Act\ one\ \{-\}} (followed by \texttt{\#\ A\ chapter})

Add an unnumbered part: \texttt{\#\ (PART\textbackslash{}*)\ Act\ one\ \{-\}} (followed by \texttt{\#\ A\ chapter})

Add an appendix as a special kind of un-numbered part: \texttt{\#\ (APPENDIX)\ Other\ stuff\ \{-\}} (followed by \texttt{\#\ A\ chapter}). Chapters in an appendix are prepended with letters instead of numbers.

\hypertarget{footnotes-and-citations}{%
\chapter{Footnotes and citations}\label{footnotes-and-citations}}

\hypertarget{footnotes}{%
\section{Footnotes}\label{footnotes}}

Footnotes are put inside the square brackets after a caret \texttt{\^{}{[}{]}}. Like this one \footnote{This is a footnote.}.

\hypertarget{citations}{%
\section{Citations}\label{citations}}

Reference items in your bibliography file(s) using \texttt{@key}.

For example, we are using the \textbf{bookdown} package \citep{R-bookdown} (check out the last code chunk in index.Rmd to see how this citation key was added) in this sample book, which was built on top of R Markdown and \textbf{knitr} \citep{xie2015} (this citation was added manually in an external file book.bib).
Note that the \texttt{.bib} files need to be listed in the index.Rmd with the YAML \texttt{bibliography} key.

The RStudio Visual Markdown Editor can also make it easier to insert citations: \url{https://rstudio.github.io/visual-markdown-editing/\#/citations}

\hypertarget{blocks}{%
\chapter{Blocks}\label{blocks}}

\hypertarget{equations}{%
\section{Equations}\label{equations}}

Here is an equation.

\begin{equation} 
  f\left(k\right) = \binom{n}{k} p^k\left(1-p\right)^{n-k}
  \label{eq:binom}
\end{equation}

You may refer to using \texttt{\textbackslash{}@ref(eq:binom)}, like see Equation \eqref{eq:binom}.

\hypertarget{theorems-and-proofs}{%
\section{Theorems and proofs}\label{theorems-and-proofs}}

Labeled theorems can be referenced in text using \texttt{\textbackslash{}@ref(thm:tri)}, for example, check out this smart theorem \ref{thm:tri}.

\begin{theorem}
\protect\hypertarget{thm:tri}{}\label{thm:tri}For a right triangle, if \(c\) denotes the \emph{length} of the hypotenuse
and \(a\) and \(b\) denote the lengths of the \textbf{other} two sides, we have
\[a^2 + b^2 = c^2\]
\end{theorem}

Read more here \url{https://bookdown.org/yihui/bookdown/markdown-extensions-by-bookdown.html}.

\hypertarget{callout-blocks}{%
\section{Callout blocks}\label{callout-blocks}}

The R Markdown Cookbook provides more help on how to use custom blocks to design your own callouts: \url{https://bookdown.org/yihui/rmarkdown-cookbook/custom-blocks.html}

\hypertarget{sharing-your-book}{%
\chapter{Sharing your book}\label{sharing-your-book}}

\hypertarget{publishing}{%
\section{Publishing}\label{publishing}}

HTML books can be published online, see: \url{https://bookdown.org/yihui/bookdown/publishing.html}

\hypertarget{pages}{%
\section{404 pages}\label{pages}}

By default, users will be directed to a 404 page if they try to access a webpage that cannot be found. If you'd like to customize your 404 page instead of using the default, you may add either a \texttt{\_404.Rmd} or \texttt{\_404.md} file to your project root and use code and/or Markdown syntax.

\hypertarget{metadata-for-sharing}{%
\section{Metadata for sharing}\label{metadata-for-sharing}}

Bookdown HTML books will provide HTML metadata for social sharing on platforms like Twitter, Facebook, and LinkedIn, using information you provide in the \texttt{index.Rmd} YAML. To setup, set the \texttt{url} for your book and the path to your \texttt{cover-image} file. Your book's \texttt{title} and \texttt{description} are also used.

This \texttt{gitbook} uses the same social sharing data across all chapters in your book- all links shared will look the same.

Specify your book's source repository on GitHub using the \texttt{edit} key under the configuration options in the \texttt{\_output.yml} file, which allows users to suggest an edit by linking to a chapter's source file.

Read more about the features of this output format here:

\url{https://pkgs.rstudio.com/bookdown/reference/gitbook.html}

Or use:

\begin{Shaded}
\begin{Highlighting}[]
\NormalTok{?bookdown}\SpecialCharTok{::}\NormalTok{gitbook}
\end{Highlighting}
\end{Shaded}


  \bibliography{books.bib,packages.bib}

\end{document}
